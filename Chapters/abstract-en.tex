%!TEX root = ../template.tex
%%%%%%%%%%%%%%%%%%%%%%%%%%%%%%%%%%%%%%%%%%%%%%%%%%%%%%%%%%%%%%%%%%%%
%% abstrac-en.tex
%% NOVA thesis document file
%%
%% Abstract in English([^%]*)º
% DR: REVIEWED (2022/03/27)
%%%%%%%%%%%%%%%%%%%%%%%%%%%%%%%%%%%%%%%%%%%%%%%%%%%%%%%%%%%%%%%%%%%%

%REVIEWED 2022/04/11
% W/ CORRECTIONS FROM PROF 30/03/2022


\typeout{NT FILE abstract-en.tex}

With the increase of the global environmental awareness and demand nowadays, the request for renewable energy sources by the companies responsible for distributing the energy itself, highly increased as well. The sources, such as wind turbines, are highly exposed to several external factors that can result in mechanical failures on their complex components. These faults lead to multiple consequences due to their failure time: loss of energy production, which means that less renewable energy will be transmitted to the electrical grid and more money will be spent on repairing these components. The content of this thesis consists of an analysis to the implementation of a prediction system allied with a monitoring system. With the data obtained from this new system, it will be possible to implement existing machine learning technologies, allowing failure detection on wind assets, to anticipate them, reducing maintenance costs and increasing the longevity of the components, which is directly linked to the increase of the profit of the energy production companies on a long term.
With the work that was made, it was implemented an generic experiment to several faults of wind turbines that created machine learning models that reliably predict these faults.

% Palavras-chave do resumo em Inglês
\begin{keywords}
Renewable energy, machine learning, renewable monitoring system, wind parks, wind turbine, failure prediction
\end{keywords} 
