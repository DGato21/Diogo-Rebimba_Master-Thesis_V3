%!TEX root = ../template.tex
%%%%%%%%%%%%%%%%%%%%%%%%%%%%%%%%%%%%%%%%%%%%%%%%%%%%%%%%%%%%%%%%%%%%
%% abstrac-pt.tex
%% NOVA thesis document file
%%
%% Abstract in Portuguese
% DR: REVIEWED (2022/03/27)
%%%%%%%%%%%%%%%%%%%%%%%%%%%%%%%%%%%%%%%%%%%%%%%%%%%%%%%%%%%%%%%%%%%%


\typeout{NT FILE abstract-pt.tex}

Com a crescente preocupação ambiental existente nos dias de hoje, fontes de energia renovável são cada vez mais procuradas pelas empresas de distribui-ção de energia. Estas fontes, como as turbinas eólicas, são expostas a diversos fatores externos e a falhas mecânicas dos seus complexos componentes. Estas falhas levam a diversas consequências devido ao tempo fora de serviço: perda de produção de energia, levando a que menos energia renovável seja transmitida para a rede elétrica e também maiores custos de reparação. O conteúdo desta tese apresenta uma análise à implementação de um sistema de predição de falhas aliado a um sistema de monitorização. Com os dados obtidos deste sistema, é possível utilizar as tecnologias de aprendizagem automática atualmente existentes e assim permitir a deteção de falhas nos ativos eólicos, antecipando as mesmas e assim reduzir todos os custos referidos anteriormente, aumentando a longevidade dos componentes, diminuindo o grau de reparações necessárias e aumentando o lucro das empresas de produção de energia.

% Palavras-chave do resumo em Português
\begin{keywords}
energias renováveis, aprendizagem automática, sistema de monitorização de renováveis, parques eólicos, turbinas eólicas, previsão de falhas
\end{keywords}
% to add an extra black line
