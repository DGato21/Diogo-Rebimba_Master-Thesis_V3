%!TEX root = ../template.tex
%%%%%%%%%%%%%%%%%%%%%%%%%%%%%%%%%%%%%%%%%%%%%%%%%%%%%%%%%%%%%%%%%%%%
%% abstrac-pt.tex
%% NOVA thesis document file
%%
%% Abstract in Portuguese
% DR: REVIEWED (2022/03/27)
%%%%%%%%%%%%%%%%%%%%%%%%%%%%%%%%%%%%%%%%%%%%%%%%%%%%%%%%%%%%%%%%%%%%

%REVIEWED 2022/04/11
% W/ CORRECTIONS FROM PROF 30/03/2022

%REVER O QUE ESTA ESCRITO PELO PROF NESTA PAGINA. NAO PERCEBI!


\typeout{NT FILE abstract-pt.tex}

Com a crescente preocupação ambiental existente nos dias de hoje, fontes de energia renovável são cada vez mais procuradas pelas empresas de distribuição de energia. Estas fontes, como as turbinas eólicas, são expostas a diversos fatores externos e a falhas mecânicas dos seus complexos componentes. Estas falhas levam a diversas consequências devido ao tempo fora de serviço: perda de produção de energia, levando a que menos energia renovável seja transmitida para a rede elétrica e também maiores custos de reparação. O conteúdo desta tese apresenta uma análise à implementação de um sistema de predição de falhas aliado a um sistema de monitorização. Este sistema irá permitir a deteção de falhas nos ativos eólicos, antecipando as mesmas e assim reduzindo todos os custos de manutenção, aumentando a longevidade dos componentes. A redução das falhas irá permitir à empresa aumentar o lucro derivado da produção de energia a longo prazo.
Com o trabalho realizado, foi possível implementar uma experiência que pode ser aplicada às diversas falhas de turbinas eólicas que crie modelos de aprendizagem automática que consigam prever estas mesmas falhas de um modo fiável.

% Palavras-chave do resumo em Português
\begin{keywords}
energias renováveis, aprendizagem automática, sistema de monitorização de renováveis, parques eólicos, turbinas eólicas, previsão de falhas
\end{keywords}
% to add an extra black line
